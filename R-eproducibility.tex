% Options for packages loaded elsewhere
\PassOptionsToPackage{unicode}{hyperref}
\PassOptionsToPackage{hyphens}{url}
%
\documentclass[
]{article}
\author{}
\date{\vspace{-2.5em}}

\usepackage{amsmath,amssymb}
\usepackage{lmodern}
\usepackage{iftex}
\ifPDFTeX
  \usepackage[T1]{fontenc}
  \usepackage[utf8]{inputenc}
  \usepackage{textcomp} % provide euro and other symbols
\else % if luatex or xetex
  \usepackage{unicode-math}
  \defaultfontfeatures{Scale=MatchLowercase}
  \defaultfontfeatures[\rmfamily]{Ligatures=TeX,Scale=1}
\fi
% Use upquote if available, for straight quotes in verbatim environments
\IfFileExists{upquote.sty}{\usepackage{upquote}}{}
\IfFileExists{microtype.sty}{% use microtype if available
  \usepackage[]{microtype}
  \UseMicrotypeSet[protrusion]{basicmath} % disable protrusion for tt fonts
}{}
\makeatletter
\@ifundefined{KOMAClassName}{% if non-KOMA class
  \IfFileExists{parskip.sty}{%
    \usepackage{parskip}
  }{% else
    \setlength{\parindent}{0pt}
    \setlength{\parskip}{6pt plus 2pt minus 1pt}}
}{% if KOMA class
  \KOMAoptions{parskip=half}}
\makeatother
\usepackage{xcolor}
\IfFileExists{xurl.sty}{\usepackage{xurl}}{} % add URL line breaks if available
\IfFileExists{bookmark.sty}{\usepackage{bookmark}}{\usepackage{hyperref}}
\hypersetup{
  hidelinks,
  pdfcreator={LaTeX via pandoc}}
\urlstyle{same} % disable monospaced font for URLs
\usepackage[margin=1in]{geometry}
\usepackage{color}
\usepackage{fancyvrb}
\newcommand{\VerbBar}{|}
\newcommand{\VERB}{\Verb[commandchars=\\\{\}]}
\DefineVerbatimEnvironment{Highlighting}{Verbatim}{commandchars=\\\{\}}
% Add ',fontsize=\small' for more characters per line
\usepackage{framed}
\definecolor{shadecolor}{RGB}{248,248,248}
\newenvironment{Shaded}{\begin{snugshade}}{\end{snugshade}}
\newcommand{\AlertTok}[1]{\textcolor[rgb]{0.94,0.16,0.16}{#1}}
\newcommand{\AnnotationTok}[1]{\textcolor[rgb]{0.56,0.35,0.01}{\textbf{\textit{#1}}}}
\newcommand{\AttributeTok}[1]{\textcolor[rgb]{0.77,0.63,0.00}{#1}}
\newcommand{\BaseNTok}[1]{\textcolor[rgb]{0.00,0.00,0.81}{#1}}
\newcommand{\BuiltInTok}[1]{#1}
\newcommand{\CharTok}[1]{\textcolor[rgb]{0.31,0.60,0.02}{#1}}
\newcommand{\CommentTok}[1]{\textcolor[rgb]{0.56,0.35,0.01}{\textit{#1}}}
\newcommand{\CommentVarTok}[1]{\textcolor[rgb]{0.56,0.35,0.01}{\textbf{\textit{#1}}}}
\newcommand{\ConstantTok}[1]{\textcolor[rgb]{0.00,0.00,0.00}{#1}}
\newcommand{\ControlFlowTok}[1]{\textcolor[rgb]{0.13,0.29,0.53}{\textbf{#1}}}
\newcommand{\DataTypeTok}[1]{\textcolor[rgb]{0.13,0.29,0.53}{#1}}
\newcommand{\DecValTok}[1]{\textcolor[rgb]{0.00,0.00,0.81}{#1}}
\newcommand{\DocumentationTok}[1]{\textcolor[rgb]{0.56,0.35,0.01}{\textbf{\textit{#1}}}}
\newcommand{\ErrorTok}[1]{\textcolor[rgb]{0.64,0.00,0.00}{\textbf{#1}}}
\newcommand{\ExtensionTok}[1]{#1}
\newcommand{\FloatTok}[1]{\textcolor[rgb]{0.00,0.00,0.81}{#1}}
\newcommand{\FunctionTok}[1]{\textcolor[rgb]{0.00,0.00,0.00}{#1}}
\newcommand{\ImportTok}[1]{#1}
\newcommand{\InformationTok}[1]{\textcolor[rgb]{0.56,0.35,0.01}{\textbf{\textit{#1}}}}
\newcommand{\KeywordTok}[1]{\textcolor[rgb]{0.13,0.29,0.53}{\textbf{#1}}}
\newcommand{\NormalTok}[1]{#1}
\newcommand{\OperatorTok}[1]{\textcolor[rgb]{0.81,0.36,0.00}{\textbf{#1}}}
\newcommand{\OtherTok}[1]{\textcolor[rgb]{0.56,0.35,0.01}{#1}}
\newcommand{\PreprocessorTok}[1]{\textcolor[rgb]{0.56,0.35,0.01}{\textit{#1}}}
\newcommand{\RegionMarkerTok}[1]{#1}
\newcommand{\SpecialCharTok}[1]{\textcolor[rgb]{0.00,0.00,0.00}{#1}}
\newcommand{\SpecialStringTok}[1]{\textcolor[rgb]{0.31,0.60,0.02}{#1}}
\newcommand{\StringTok}[1]{\textcolor[rgb]{0.31,0.60,0.02}{#1}}
\newcommand{\VariableTok}[1]{\textcolor[rgb]{0.00,0.00,0.00}{#1}}
\newcommand{\VerbatimStringTok}[1]{\textcolor[rgb]{0.31,0.60,0.02}{#1}}
\newcommand{\WarningTok}[1]{\textcolor[rgb]{0.56,0.35,0.01}{\textbf{\textit{#1}}}}
\usepackage{graphicx}
\makeatletter
\def\maxwidth{\ifdim\Gin@nat@width>\linewidth\linewidth\else\Gin@nat@width\fi}
\def\maxheight{\ifdim\Gin@nat@height>\textheight\textheight\else\Gin@nat@height\fi}
\makeatother
% Scale images if necessary, so that they will not overflow the page
% margins by default, and it is still possible to overwrite the defaults
% using explicit options in \includegraphics[width, height, ...]{}
\setkeys{Gin}{width=\maxwidth,height=\maxheight,keepaspectratio}
% Set default figure placement to htbp
\makeatletter
\def\fps@figure{htbp}
\makeatother
\setlength{\emergencystretch}{3em} % prevent overfull lines
\providecommand{\tightlist}{%
  \setlength{\itemsep}{0pt}\setlength{\parskip}{0pt}}
\setcounter{secnumdepth}{-\maxdimen} % remove section numbering
\ifLuaTeX
  \usepackage{selnolig}  % disable illegal ligatures
\fi

\begin{document}

\hypertarget{project-info}{%
\subsection{\#\#\# Project Info}\label{project-info}}

title: ``R-eproducibility'' author: ``mstewart2000'' date:
``26/01/2022'' output: html\_document GitHub Link:
\url{https://github.com/mstewart2000/R-eproducibility/tree/main} ---

\hypertarget{initial-setup}{%
\subsection{Initial Setup}\label{initial-setup}}

\hypertarget{loading-packages-and-setting-up-working-directory}{%
\section{Loading packages and setting up working
directory}\label{loading-packages-and-setting-up-working-directory}}

\begin{verbatim}
## 
## Attaching package: 'dplyr'
\end{verbatim}

\begin{verbatim}
## The following objects are masked from 'package:stats':
## 
##     filter, lag
\end{verbatim}

\begin{verbatim}
## The following objects are masked from 'package:base':
## 
##     intersect, setdiff, setequal, union
\end{verbatim}

\begin{verbatim}
## [1] "C:/Users/malco/Documents/School/2021-22/BIOL 432/Week 3/R-eproducibility"
\end{verbatim}

\hypertarget{load-data-from-folder-and-check-integrity}{%
\section{Load data from folder and check
integrity}\label{load-data-from-folder-and-check-integrity}}

\begin{verbatim}
##   PotNum Scenario Nutrients Taxon Symphytum Silene Urtica Geranium Geum
## 1      1      low       low japon      9.81  36.36  16.08     4.68 0.12
## 2      2      low       low japon      8.64  29.65   5.59     5.75 0.55
## 3      3      low       low japon      2.65  36.03  17.09     5.13 0.09
## 4      5      low       low japon      1.44  21.43  12.39     5.37 0.31
## 5      6      low       low japon      9.15  23.90   5.19     0.00 0.17
## 6      7      low       low japon      6.31  24.40   7.00     9.05 0.97
##   All_Natives Fallopia Total Pct_Fallopia
## 1       67.05     0.01 67.06         0.01
## 2       50.18     0.04 50.22         0.08
## 3       60.99     0.09 61.08         0.15
## 4       40.94     0.77 41.71         1.85
## 5       38.41     3.40 41.81         8.13
## 6       47.73     0.54 48.27         1.12
\end{verbatim}

\begin{verbatim}
## 'data.frame':    123 obs. of  13 variables:
##  $ PotNum      : int  1 2 3 5 6 7 8 9 10 11 ...
##  $ Scenario    : chr  "low" "low" "low" "low" ...
##  $ Nutrients   : chr  "low" "low" "low" "low" ...
##  $ Taxon       : chr  "japon" "japon" "japon" "japon" ...
##  $ Symphytum   : num  9.81 8.64 2.65 1.44 9.15 ...
##  $ Silene      : num  36.4 29.6 36 21.4 23.9 ...
##  $ Urtica      : num  16.08 5.59 17.09 12.39 5.19 ...
##  $ Geranium    : num  4.68 5.75 5.13 5.37 0 9.05 3.51 9.64 7.3 6.36 ...
##  $ Geum        : num  0.12 0.55 0.09 0.31 0.17 0.97 0.4 0.01 0.47 0.33 ...
##  $ All_Natives : num  67 50.2 61 40.9 38.4 ...
##  $ Fallopia    : num  0.01 0.04 0.09 0.77 3.4 0.54 2.05 0.26 0 0 ...
##  $ Total       : num  67.1 50.2 61.1 41.7 41.8 ...
##  $ Pct_Fallopia: num  0.01 0.08 0.15 1.85 8.13 1.12 3.7 0.61 0 0 ...
\end{verbatim}

\hypertarget{reorganize-data}{%
\subsection{Reorganize Data}\label{reorganize-data}}

\hypertarget{remove-rows-with-biomass-60}{%
\section{Remove rows with Biomass \textgreater{}
60}\label{remove-rows-with-biomass-60}}

\hypertarget{rearrange-collums-and-only-keep-total-taxon-scenario-and-nutrients}{%
\section{Rearrange Collums and only keep Total, Taxon, Scenario, and
Nutrients}\label{rearrange-collums-and-only-keep-total-taxon-scenario-and-nutrients}}

\begin{Shaded}
\begin{Highlighting}[]
\NormalTok{data }\OtherTok{=}\NormalTok{ data }\SpecialCharTok{\%\textgreater{}\%}
  \FunctionTok{select}\NormalTok{(Total, Taxon, Scenario, Nutrients) }\SpecialCharTok{\%\textgreater{}\%}
  \FunctionTok{arrange}\NormalTok{(Total, Taxon, Scenario, Nutrients)}
\end{Highlighting}
\end{Shaded}

\hypertarget{create-a-new-collum-which-is-total-biomass-in-grams}{%
\section{Create a new collum which is Total Biomass in
Grams}\label{create-a-new-collum-which-is-total-biomass-in-grams}}

\begin{Shaded}
\begin{Highlighting}[]
\NormalTok{data }\OtherTok{=}\NormalTok{ data }\SpecialCharTok{\%\textgreater{}\%}
  \FunctionTok{mutate}\NormalTok{(}\AttributeTok{TotalG =}\NormalTok{ Total}\SpecialCharTok{/}\DecValTok{1000}\NormalTok{) }\SpecialCharTok{\%\textgreater{}\%}
  \FunctionTok{select}\NormalTok{(TotalG, Taxon, Scenario, Nutrients)}
\end{Highlighting}
\end{Shaded}

\hypertarget{custom-function}{%
\subsection{Custom Function}\label{custom-function}}

\hypertarget{this-function-will-allow-you-to-find-the-average-sum-or-number-of-observations-for-a-selected-column}{%
\section{This function will allow you to find the average, sum, or
number of observations for a selected
column}\label{this-function-will-allow-you-to-find-the-average-sum-or-number-of-observations-for-a-selected-column}}

\hypertarget{please-follow-the-following-input-method---data.framecolumn.name-argument}{%
\section{Please follow the following input method -\textgreater{}
(data.frame\$column.name,
``argument'')}\label{please-follow-the-following-input-method---data.framecolumn.name-argument}}

\hypertarget{possible-arguments-are-average-sum-observations}{%
\section{Possible arguments are: Average, Sum,
Observations}\label{possible-arguments-are-average-sum-observations}}

\begin{Shaded}
\begin{Highlighting}[]
\NormalTok{DataInformation }\OtherTok{\textless{}{-}} \ControlFlowTok{function}\NormalTok{(x, y) \{}
\NormalTok{  t }\OtherTok{\textless{}{-}} \ControlFlowTok{if}\NormalTok{ (y }\SpecialCharTok{==} \StringTok{"Average"}\NormalTok{) \{}
    \FunctionTok{print}\NormalTok{(}\FunctionTok{paste}\NormalTok{(}\StringTok{"The average of the selected column is:"}\NormalTok{, }\FunctionTok{mean}\NormalTok{(x)))}
\NormalTok{  \} }\ControlFlowTok{else} \ControlFlowTok{if}\NormalTok{ (y }\SpecialCharTok{==} \StringTok{"Sum"}\NormalTok{) \{}
    \FunctionTok{print}\NormalTok{(}\FunctionTok{paste}\NormalTok{(}\StringTok{"The sum of the selected column is:"}\NormalTok{, }\FunctionTok{sum}\NormalTok{(x)))}
\NormalTok{  \} }\ControlFlowTok{else} \ControlFlowTok{if}\NormalTok{ (y }\SpecialCharTok{==} \StringTok{"Observations"}\NormalTok{) \{}
    \FunctionTok{print}\NormalTok{(}\FunctionTok{paste}\NormalTok{(}\StringTok{"The length of the selected column is:"}\NormalTok{, }\FunctionTok{length}\NormalTok{(x)))}
\NormalTok{  \} }\ControlFlowTok{else}\NormalTok{ \{}
    \FunctionTok{print}\NormalTok{(}\FunctionTok{paste}\NormalTok{(}\StringTok{"Error: please ensure that the selected column exists and that you have chosen either Average, Sum, or Observations"}\NormalTok{))}
\NormalTok{  \}}
\NormalTok{\}}
\end{Highlighting}
\end{Shaded}

\hypertarget{testing-the-function}{%
\section{Testing the Function}\label{testing-the-function}}

\begin{Shaded}
\begin{Highlighting}[]
\FunctionTok{DataInformation}\NormalTok{(data}\SpecialCharTok{$}\NormalTok{Taxon, }\StringTok{"Observations"}\NormalTok{)}
\end{Highlighting}
\end{Shaded}

\begin{verbatim}
## [1] "The length of the selected column is: 78"
\end{verbatim}

\begin{Shaded}
\begin{Highlighting}[]
\FunctionTok{DataInformation}\NormalTok{(}\FunctionTok{subset}\NormalTok{(data}\SpecialCharTok{$}\NormalTok{TotalG, data}\SpecialCharTok{$}\NormalTok{Nutrients }\SpecialCharTok{==} \StringTok{"high"}\NormalTok{), }\StringTok{"Average"}\NormalTok{); }\FunctionTok{DataInformation}\NormalTok{(}\FunctionTok{subset}\NormalTok{(data}\SpecialCharTok{$}\NormalTok{TotalG, data}\SpecialCharTok{$}\NormalTok{Nutrients }\SpecialCharTok{==} \StringTok{"low"}\NormalTok{), }\StringTok{"Average"}\NormalTok{) }
\end{Highlighting}
\end{Shaded}

\begin{verbatim}
## [1] "The average of the selected column is: 0.0513247272727273"
\end{verbatim}

\begin{verbatim}
## [1] "The average of the selected column is: 0.0486939130434783"
\end{verbatim}

\hypertarget{knitting-the-data-to-the-ouput-file}{%
\subsection{Knitting the Data to the Ouput
File}\label{knitting-the-data-to-the-ouput-file}}

\begin{Shaded}
\begin{Highlighting}[]
\FunctionTok{write.csv}\NormalTok{(data, }\StringTok{"Output/WrangledData.csv"}\NormalTok{)}
\end{Highlighting}
\end{Shaded}


\end{document}
